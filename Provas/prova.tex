\documentclass[12pt]{exam}

% Básico de Idioma
\usepackage[utf8]{inputenc} 
\usepackage[T1]{fontenc}
\usepackage[portuguese]{babel}

% Matemática
\usepackage{amsmath}
\usepackage{amsfonts}
\usepackage{amssymb}
\usepackage{amsthm}
\usepackage{mathtools}
\usepackage{centernot}

% Cores
\usepackage{color}
\usepackage{xcolor}

% Desenho
\usepackage{tikz}
\usepackage{tikz-qtree}
\usetikzlibrary{positioning, calc, chains, fit, shapes, automata, trees}

% Geometria do documento
\usepackage[margin=1in]{geometry}

% Multi colunas
\usepackage{multicol}

% Itens nas lista
\usepackage[shortlabels]{enumitem}

% Novos comandos
\newcommand{\university}[1]{\textbf{#1}}
\newcommand{\colleage}[1]{\textbf{#1}}
\newcommand{\course}[2]{\textbf{#1} --- \textbf{#2}}
\newcommand{\period}[2]{\textbf{Semestre: #1}.\textbf{#2}} 
\newcommand{\teacher}[1]{\textbf{Professor: #1}}
\newcommand{\examdate}[3]{\textit{#1/#2/#3}}
%\newcommand{\timelimit}{}          


% O documento
\begin{document}
	\pagestyle{plain}
	\thispagestyle{empty}
	\noindent
	% Construção do cabeçálho
	\begin{tabular*}{\textwidth}{ll}
		\university{Universidade da Cidade de Arkhan}\\
		\colleage{Departamento de Ciências}\\
		\course{cth0100}{Introdução aos Mythos}\\
		\period{1916}{1}\\
		\teacher{H. P. Lovercraft}\\
		\textbf{Atividade Avaliativa} --- \examdate{01}{06}{1916}\\
	\end{tabular*}\\
	% Passando a linha
	\rule[0.1ex]{\textwidth}{1pt}
	\begin{center}
		Recomendações ({\color{red} leia com atenção})
	\end{center}
	\begin{itemize}
		\item Primeira recomendação.
		\item Segunda recomendação.
		\item Terceira recomendação.
		\item Quarta recomendação.
		\item Quinta recomendação.
	\end{itemize}
	% Passando a linha
	\rule[0.1ex]{\textwidth}{1pt}
	\begin{center}
		{\LARGE Questionário}
	\end{center}
	\begin{questions}
		\question ({\color{blue} 2.5 pontos}) Quem é o autor do necronomicon?
		\question ({\color{blue} 2.5 pontos}) Quem é Nyarlathotep?
		\question ({\color{blue} 2.5 pontos}) Qual o nome da cidade submersa do deus Cthulhu?
		\question ({\color{blue} 2.5 pontos}) Qual o nome do marinho que encontrou a ilha do Dagon?
	\end{questions}
\end{document}      